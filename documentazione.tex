%%%%%%%%%%%%%%%%%%%%%%%%%%%%%%%%%%%%%%%%%
% University/School Laboratory Report
% LaTeX Template
% Version 3.1 (25/3/14)
%
% This template has been downloaded from:
% http://www.LaTeXTemplates.com
%
% Original author:
% Linux and Unix Users Group at Virginia Tech Wiki
% (https://vtluug.org/wiki/Example_LaTeX_chem_lab_report)
%
% License:
% CC BY-NC-SA 3.0 (http://creativecommons.org/licenses/by-nc-sa/3.0/)
%
%%%%%%%%%%%%%%%%%%%%%%%%%%%%%%%%%%%%%%%%%

%----------------------------------------------------------------------------------------
%	PACKAGES AND DOCUMENT CONFIGURATIONS
%----------------------------------------------------------------------------------------

\documentclass{article}
\usepackage[utf8]{inputenc}
\usepackage{appendix}
\usepackage[T1]{fontenc}
\usepackage{siunitx} % Provides the \SI{}{} and \si{} command for typesetting SI units
\usepackage{graphicx} % Required for the inclusion of images
\usepackage{natbib} % Required to change bibliography style to APA
\usepackage{amsmath} % Required for some math elements
\usepackage{caption}
\usepackage{tikz}

\usetikzlibrary{arrows,automata, positioning}

\usepackage{import}

\setlength\parindent{0pt} % Removes all indentation from paragraphs

%\renewcommand{\labelenumi}{\alph{enumi}.} % Make numbering in the enumerate environment by letter rather than number (e.g. section 6)

%\usepackage{times} % Uncomment to use the Times New Roman font

%----------------------------------------------------------------------------------------
%	DOCUMENT INFORMATION
%----------------------------------------------------------------------------------------

\title{Prova Finale di Reti Logiche} % Title
\author{Truong Kien Tuong} % Author name
\date{May 1st, 2020}

\begin{document}
\maketitle % Insert the title, author and date
\begin{center}
\begin{tabular}{l r}
Matricola: & 887907\\ % Partner names
Codice Persona: & 10582491\\
Docente: & Gianluca Palermo % Instructor/supervisor
\end{tabular}
\end{center}

% If you wish to include an abstract, uncomment the lines below
% \begin{abstract}
% Abstract text
% \end{abstract}

%----------------------------------------------------------------------------------------
%	SECTION 1
%----------------------------------------------------------------------------------------

\section{Introduzione}

La prova prevede l'implementazione in VHDL di una macchine che opera su una memoria e svolge la seguente operazione.

La macchina deve per prima cosa leggere il primo byte dalla memoria che identifica il numero di parole che sono
state fornite come input, questa informazione è importante per capire quando la macchina deve terminare la lettura.

Dopodiché ogni parola successiva viene tradotta in due parole di memoria che vengono scritte progressivamente a partire
dal millesimo indirizzo di memoria.

Esempio:



%----------------------------------------------------------------------------------------
%	SECTION 2
%----------------------------------------------------------------------------------------

\section{Architetttura}
Architetttura



%----------------------------------------------------------------------------------------
%	SECTION 3
%----------------------------------------------------------------------------------------

\section{Risultati sperimentali}
Risultati sperimentali

%----------------------------------------------------------------------------------------
%	SECTION 4
%----------------------------------------------------------------------------------------

\section{Conclusioni}

Conclusioni

%----------------------------------------------------------------------------------------

\end{document}